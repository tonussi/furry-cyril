\documentclass[conference]{IEEEtran}
\usepackage{cite, amsmath, graphicx, hyperref, algorithm}
\usepackage[utf8]{inputenc}
\usepackage[brazil, english]{babel}
\usepackage[noend]{algpseudocode}

\title {
$(\textit{Miller-Rabin}\text{ Vs }(\frac{\textit{Solovay-Strassen}}{
  \textit{Adrien-Legendre}}))^{\frac{Mersenne-Twister}{Linux-Entropy}}$
}

\author {
  Lucas Tonussi\\
  \textsc{lptonussi@gmail.com}\\
  \small{Ciência da Computação}
}

\begin{document}

\maketitle

\begin{abstract}
Esse artigo trás conclusões sobre a comparação entre \textit{Miller-Rabin} e
\textit{Solovay-Strassen} como algoritmos de teste de número primos, impares e
pseudo-aleatórios calculados através do algoritmo \textit{Mersenne-Twister}
usando Entropia Linux.
\end{abstract}

\IEEEoverridecommandlockouts

\begin{keywords}
Teste de números primos, Miller-Rabin, Solovay-Strassen, Mersenne-Twister, 
Adrien-Legendre, Números Pseudo Aleatórios, Entropia-Linux
\end{keywords}

\IEEEpeerreviewmaketitle

\section{Introdução}

\section{Algoritmo Solovay Strassen}

Esse algoritmo trás uma perspectiva mais robusta e simples para verificar se um
número ímpar é primo ou não (composto). A diferença no algorítimo de Solovay
Strassen é o uso de um algoritmo interno recursivo bastante parecido do MCD
euclidiano. Que é a notação $(\frac{a}{n})$ onde \textbf{a} é um aleatório como
visto abaixo e n é um primo. Essa notação exige um processo recursivo para
derivar modularmente. Em Python a solução estão nas linhas 35 à 46 que é o
algoritmo de Adrien-Legendre que consta na wikipédia sobre o algoritmo Solovay
Strassen. Para números grande e muitas iterações pegando valores aleatórios para
Adrien-Legendre o algoritmo de Solovay-Strassen se sai melhor. Porém no geral
para números grandes ambos utilizam a técnica bastante parecida de base com
números aleatórios tentando prever um candidato. Ambos tem um custo temporal
equivalente no geral, para um mesmo primo ambos e com acuracidade igual de 25
iterações no máximo, ambos tiveram o mesmo desempenho e taxas de rejeições e
taxas de acertos em qualificação de números primos, você pode executar o código
utilizando \textbf{python} $\_\_init\_\_.py$ ou \textbf{python3}
$\_\_init\_\_.py$. Ele criará um arquivo chamado entropy\cite{dd} e outro
chamado stderr o primeiro contêm dados aleatórios providos pelo
/dev/urandom\cite{dd} no Unix. Veja também o arquivo DATA que é a saída da minha
execução do código em python em ambiente Linux.

\begin{algorithm}
  \caption{Algoritmo de Solovay Strassen}
  \begin{algorithmic}
    \Function{$Solovay Strassen$}{}
      \vfil
      \State {$primo \gets (Valor > 2)$}
      \vfil
      \State {$acuracidade \gets (Valor > 5)$}
      \vfil
      \While{$it \le k$}
        \vfil
        \State $a \gets \textit{Valor aleatório entre }
               (2, \textit{ primo } - 1)$
        \vfil
        \State $x \gets \left(\frac{a}{primo} \right) \left( \frac{a_i}{primo}
                  \right) \dots \left( \frac{a \cdot a_i}{primo}\right)$
        \vfil
        \State {$it \gets it + 1$}
        \vfil
      \EndWhile
      \vfil
      \If {$x = 0 \textbf{ or } a^{(n-1)/2} \not \equiv x \mod n$}
        \State \Return \textit{Falso}
      \EndIf
      \vfil
      \State \Return \textit{Verdadeiro}
    \EndFunction
  \end{algorithmic}
\end{algorithm}

\section{Conclusões}

A conclusão que eu chego é que Solovay-Strassen tem desempenho equivalente porém
a sua taxa de precisão aumenta bastante quanto mais números aleatórios são
testados como base para o algoritmo embutido do ADrien-Legendre. Mas ambos
tiveram desempenho equivalente no sentido de taxa de rejeições e qualificações
de primalidade. Para ambos eu utilizei o algoritmo de Mersenne
Twister\cite{mersenne} com semente cortando uma parte gerado de
\textit{/dev/urandom} a copia foi feita com o comando dd

\begin{thebibliography}{1}
  \bibitem{strassen} Solovay, Robert M.; Strassen, Volker (1977). "A fast
  Monte-Carlo test for primality". Disponível em:
  https://en.wikipedia.org/wiki/Solovay-Strassen\_primality\_test
  \bibitem{mersenne} Twister, Mersenne "Pseudo código do algoritmo gerador de
  pseudo aleatórios Mersenne Twister". Disponível em:
  https://en.wikipedia.org/wiki/Mersenne\_Twister
  \bibitem{dd} Linux/Unix, "Uso do comando unix \textit{dd} para /dev/urandom
  como semente". Disponível em: http://linux.die.net/man/1/dd
\end{thebibliography}

\smallskip

\end{document}
